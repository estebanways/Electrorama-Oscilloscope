\documentclass{article}
\usepackage{amsmath}

\title{Manufacturing and use calculations of the Electrorama oscilloscope}
\author{}
\date{}

\begin{document}
	
	\maketitle
	
	\section*{Content}
	\begin{itemize}
		\item Voltage Divider Current Calculation
		\item Designing a Voltage Divider for High-Voltage Signal Measurement: Effects on Frequency, Waveform, and Safety
	\end{itemize}
	
	\section{Voltage Divider Current Calculation}
	
	\subsection*{I made a voltage divider, from 120 V 60 Hz, to 2.2 V using 1/4 W resistors. What would be the output in \textbf{amps}?}

	To calculate the output current in your voltage divider circuit, we need to know the total resistance of the circuit and apply Ohm's Law:
	
	\[
	I = \frac{V}{R}
	\]
	
	Where:
	\begin{itemize}
		\item \( I \) is the current in amps.
		\item \( V \) is the output voltage in volts (2.2 V in your case).
		\item \( R \) is the total resistance in ohms.
	\end{itemize}
	
	Since you're using 1/4 W resistors, the power rating limits the maximum allowable current to avoid overheating the resistors. For a 1/4 W resistor, the maximum current \( I_{\text{max}} \) can be calculated as:
	
	\[
	I_{\text{max}} = \sqrt{\frac{P}{R}}
	\]
	
	Where:
	\begin{itemize}
		\item \( P = 0.25 \, \text{W} \) (power rating of the resistor),
		\item \( R \) is the resistance of each resistor.
	\end{itemize}
	
	Now, to determine the actual current output:
	\begin{enumerate}
		\item What is the resistance of each resistor?
		\item How are the resistors arranged (in series or parallel)?
	\end{enumerate}
	
	Example:
	\( R_1 = 5.5 \, \text{M}\Omega \), \( R_2 = 100 \, \text{k}\Omega \)
	
	Let's calculate the output current for your voltage divider with:
	\begin{itemize}
		\item \( R_1 = 5.5 \, \text{M}\Omega = 5,500,000 \, \Omega \)
		\item \( R_2 = 100 \, \text{k}\Omega = 100,000 \, \Omega \)
		\item Input voltage (\( V_{\text{in}} \)) = 120 V
		\item Output voltage (\( V_{\text{out}} \)) = 2.2 V (measured across \( R_2 \)).
	\end{itemize}
	
	\subsubsection*{Step 1: Verify the voltage divider output}
	
	The voltage across \( R_2 \) in a voltage divider is calculated using the formula:
	
	\[
	V_{\text{out}} = V_{\text{in}} \cdot \frac{R_2}{R_1 + R_2}
	\]
	
	Substitute the values:
	
	\[
	V_{\text{out}} = 120 \cdot \frac{100,000}{5,500,000 + 100,000} 
	\]
	
	\[
	V_{\text{out}} = 120 \cdot \frac{100,000}{5,600,000} 
	\]
	
	\[
	V_{\text{out}} = 120 \cdot 0.017857 \approx 2.1428 \, \text{V}
	\]
	
	This is close to your measured output (2.2 V), confirming the setup.
	
	\subsubsection*{Step 2: Calculate the current}
	
	The current in the circuit is the same through both resistors because they are in series. Using Ohm's Law:
	
	\[
	I = \frac{V_{\text{in}}}{R_1 + R_2}
	\]
	
	Substitute the values:
	
	\[
	I = \frac{120}{5,500,000 + 100,000} = \frac{120}{5,600,000}
	\]
	
	\[
	I = 0.00002143 \, \text{A} = 21.43 \, \mu\text{A}
	\]
	
	\subsubsection*{Final Answer}
	
	The output current is approximately \textbf{21.43 \, \mu\text{A}}.

\subsection*{Topic}

And what is the value of \( I_{\text{max}} \)?

To calculate \( I_{\text{max}} \), the maximum current the resistors can handle before exceeding their power rating, we use the formula:

\[
I_{\text{max}} = \sqrt{\frac{P}{R}}
\]

Where:
\begin{itemize}
	\item \( P = 0.25 \, \text{W} \) (power rating of the resistor),
	\item \( R \) is the resistance of the resistor.
\end{itemize}

For \( R_1 = 5.5 \, \text{M}\Omega \):

\[
I_{\text{max}} = \sqrt{\frac{0.25}{5,500,000}} 
\]

\[
I_{\text{max}} = \sqrt{4.545 \times 10^{-8}} 
\]

\[
I_{\text{max}} \approx 6.74 \, \mu\text{A}
\]

For \( R_2 = 100 \, \text{k}\Omega \):

\[
I_{\text{max}} = \sqrt{\frac{0.25}{100,000}} 
\]

\[
I_{\text{max}} = \sqrt{2.5 \times 10^{-6}} 
\]

\[
I_{\text{max}} \approx 1.58 \, \text{mA}
\]

\subsubsection*{Final Values}

\begin{itemize}
	\item \( I_{\text{max}} \) for \( R_1 \): \textbf{6.74 \, \mu\text{A}}

	\item \( I_{\text{max}} \) for \( R_2 \): \textbf{1.58 \, \text{mA}}
\end{itemize}

Since both resistors are in series, the overall \( I_{\text{max}} \) is limited by the smaller value:

\[
I_{\text{max (overall)}} = 6.74 \, \mu\text{A}
\]

This is the maximum safe current for your circuit.




\end{document}
